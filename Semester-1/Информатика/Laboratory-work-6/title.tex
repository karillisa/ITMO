\documentclass[twocolumn,4pt]{article}
\usepackage[left=15mm, top=10mm, right=15mm, bottom=10mm]{geometry} %Управление полями страницы
\usepackage[utf8x]{inputenc}    % Включаем поддержку UTF8 
\usepackage[russian]{babel}     % Включаем пакет для поддержки русского языка 
\usepackage{graphicx}   % Включаем файлы с рисунками
\usepackage{amsmath}    %добавляем поддержку расширенных математических возможностей
\usepackage{setspace} %для изменения межстрочного интервала в документе
\usepackage{tikz} %для включения библиотеки TikZ, которая позволяет создавать векторные графические изображения
\usetikzlibrary{automata,positioning} %для загрузки дополнительных пакетов TikZ
\geometry{verbose,tmargin=1in,bmargin=1in,lmagin=1.1in,rmargin=1.1in} %для настройки параметров страницы. Параметры tmargin, bmargin, lmargin и rmargin задают размеры верхнего, нижнего, левого и правого полей
\title{}

\begin{document}
\thispagestyle{empty}
\begin{titlepage}
\centering

\vspace*{\fill} %для добавления вертикального пространства
{\large Федеральное государственное автономное образовательное учреждение высшего образования «Санкт-Петербургский национальный исследовательский университет информационных технологий, механики и оптики»} \\ [0.5cm]
{\large \textbf{Факультет программной инженерии и компьютерной техники}} \\[1.75cm]
{\Huge Лаборатоная работа № 6} \\ [2mm]
{\Huge по информатике} \\ [1.5mm]
{\Large Тема: «Работа с системой компьютерной вёрстки TEX} \\ [1.5cm]
{\Large Сдала: Гафурова Фарангиз Фуркатовна} \\ [1.5mm]
{\Large \emph{Приняла: Болдырева Елена Александровна}} \\ [3.0cm]
{\large г. Санкт-Петербург. 2023г}

\vfill
\end{titlepage}

\newpage
\setstretch{0.65}
Вариант 2

\begin{enumerate}
    \item $x=h-gh^2/(2v_0^2)=15,1$ м.
    \item $a=g(m_2-m_1)/(m_1+m_2)=5$ м/с$^2$.
    \item $F=A/(s \cos \alpha)=50$ Н.
    \item $l=m^2v^2/(2 \mu g M^2=2$ м.
    \item $V=vRT/p=9,3$ м$^3$, где $R=8,31$ Дж/(моль*К) -- универсальная газовая постоянная.
    \item $r=m(c \delta t + \lambda)/P=5474$ с.
    \item $q=S*R/(R+r)=10$ мкФ.
    \item $F=q^2/(4 \pi \epsilon_0 r^2=9*10^3$ Н.
    \item $m_1/m_2=(r_1/r_2)^2=4$.
    \item $\alpha = 30^o$. \\
\end{enumerate}

\textbf{Калейдоскоп "Кванта"} \\
"Квант" № 2 \\

Вопросы и задачи 

\begin{enumerate}
    \item Свет, испускаемый лазером, -- почти строго параллельные лучи. 
    \item Для разных длин световых воли показатели преломления вещества различны. 
    \item Ближе к перпендикуляру -- красный луч, дальше всех -- фиолетовый. 
    \item Для любой линзы главное фокусное расстояние больше (по модулю) для красных лючей. 
    \item Зеленое. 
    \item Красный, поскольку при переходе из одной среды в другую частота света, определяющая цвет лучей, не измеяется. 
    \item Нет, поскольку сама интерференция -- следствие принципа суперпозиции, согласно которому фронты волн, "проникающих" одна в другую, взаимно не деформируются. 
    \item Да, так как прямая и обратная волны когерентны. 
    \item Из-за стекания воды нижняя часть пленки утолщается, а верхняя становится тоньше. Поэтому соответствующие интерференционные полосы смещаются. 
    \item Из-за дифракции на краях Луны на поверхности Земли появляется интерференционная картина. 
    \item 12. Начинают сказываться дифракционные явления. 
\end{enumerate}

Микроопыт \\
В щель будут видны темные дифракционные полосы: четкая полоса в центре и ряд более слабых боковых. \\

\textbf{"Квант" для младших школьников} \\
"Квант" № 2 
\begin{enumerate}
    \item На весах 300 монет.
    \item См. рис. 3. Сумма $S$ чисел на каждой окружности равна 12, так как $4S=36+S$.
    \item См. рис. 4.
    \item Ошибка в графе "Разность мячей" у команды Швеции: при одном выигрыше и одной ничьей разность мячей не может быть "1--1". Общее количество забитых мячей равно 11, а число пропущенных -- 12. Поэтому ошибка в счете на 1 мяч, т. е. разность мячей Швеции равна "2--1", либо "1--0". Рассмотрение этих вариантов приводит к следующей таблице:
    

  \begin{figure}
\centering
\includegraphics[width=0.5\linewidth]{F_1.png}
\label{fig:frog}
\end{figure}

\begin{table}[h!]
\centering
\begin{tabular}{|c|r|l|d|g|f|e|w|}
\hline
& Венг. & Швец. & Исп. & Ирл. & Франц. & Разн.&Очки  \\
\hline
Венгрия&*&-&-&2:1&2:0&4-1&4  \\
Швеция&-&*&1:1&1:0&-&2-1&3 \\
Испания&-&1:1&*&2:2&-&3-3&2 \\
Ирландия&1:2&0:1&2:2&*&-&3-5&1 \\
Франция&0:2&-&-&-&*&0-2&0 \\
\hline
\end{tabular}
\end{table}

\item Разрежем четырехугольник по средним линиям и сложим полученные четырехугольники так, чтобы вершины большого четырехугольника.
\end{enumerate} \\ 

{\fontsize{32}{38}\selectfont \textbf{Анкета 3-89}}  \\

\textbf{Дорогой читатель!} \\
Ежегодно в последнем номере журнала мы помещали "Нашу анкету". Но нам пришло в голову, что легче, проше высказать свое мнение, что называется, по свежим следам. Поэтому мы решили помещать анкету раз в квартал. \\
Мы обращаемся к Вам с просьбой. Ответьте, пожалуйста, на вопросы анкеты (на те, на которые Вы хотите и можете ответить), вырежьте анкету и пришлите в редакцию; на конверте напишите "АНКЕТА 3-89". \\
Очень надеемся на обратную связь. \\
\begin{enumerate}
    \item Класс, в котором Вы учитесь: \underline{\phantom{a}}\underline{\phantom{a}}\underline{\phantom{a}}\underline{\phantom{a}}\underline{\phantom{a}}\underline{\phantom{a}}\underline{\phantom{a}}\underline{\phantom{a}}\underline{\phantom{a}}\underline{\phantom{a}}\underline{\phantom{a}}\underline{\phantom{a}}\underline{\phantom{a}}\underline{\phantom{a}}\underline{\phantom{a}}\underline{\phantom{a}} \\
    Ваша профессия (если Вы работаете): \underline{\phantom{a}}\underline{\phantom{a}}\underline{\phantom{a}}\underline{\phantom{a}}\underline{\phantom{a}}\underline{\phantom{a}}\underline{\phantom{a}} \underline{\phantom{a}}\underline{\phantom{a}}\underline{\phantom{a}}\underline{\phantom{a}}\underline{\phantom{a}}\underline{\phantom{a}}\underline{\phantom{a}}\underline{\phantom{a}}\underline{\phantom{a}}\underline{\phantom{a}}\underline{\phantom{a}}\underline{\phantom{a}}\underline{\phantom{a}}\underline{\phantom{a}}\underline{\phantom{a}}\underline{\phantom{a}}\underline{\phantom{a}}\underline{\phantom{a}}\underline{\phantom{a}}\underline{\phantom{a}}\underline{\phantom{a}}\underline{\phantom{a}}\underline{\phantom{a}}\underline{\phantom{a}}\underline{\phantom{a}}\underline{\phantom{a}}\underline{\phantom{a}}\underline{\phantom{a}}\underline{\phantom{a}}\underline{\phantom{a}}\underline{\phantom{a}}\underline{\phantom{a}}\underline{\phantom{a}}\underline{\phantom{a}}\underline{\phantom{a}}\underline{\phantom{a}}\underline{\phantom{a}}\underline{\phantom{a}}\underline{\phantom{a}}\underline{\phantom{a}}\underline{\phantom{a}}\underline{\phantom{a}}\underline{\phantom{a}}\underline{\phantom{a}} \\
    круг Ваших интересов: физика, математика, астрономия, космонавтика, информатика (подчекните).
    
    \item Какие разделы журнала для Вас наиболее интересны? 
 \underline{\phantom{a}}\underline{\phantom{a}}\underline{\phantom{a}}\underline{\phantom{a}}\underline{\phantom{a}}\underline{\phantom{a}}\underline{\phantom{a}}\underline{\phantom{a}}\underline{\phantom{a}}\underline{\phantom{a}}\underline{\phantom{a}}\underline{\phantom{a}}\underline{\phantom{a}}\underline{\phantom{a}}\underline{\phantom{a}}\underline{\phantom{a}}\underline{\phantom{a}}\underline{\phantom{a}}\underline{\phantom{a}}\underline{\phantom{a}} \underline{\phantom{a}}\underline{\phantom{a}}\underline{\phantom{a}}\underline{\phantom{a}}\underline{\phantom{a}}\underline{\phantom{a}}\underline{\phantom{a}}\underline{\phantom{a}}\underline{\phantom{a}}\underline{\phantom{a}}\underline{\phantom{a}}\underline{\phantom{a}}\underline{\phantom{a}}\underline{\phantom{a}}\underline{\phantom{a}}\underline{\phantom{a}}\underline{\phantom{a}}\underline{\phantom{a}}\underline{\phantom{a}}\underline{\phantom{a}}\underline{\phantom{a}}\underline{\phantom{a}}\underline{\phantom{a}}\underline{\phantom{a}}\underline{\phantom{a}}\underline{\phantom{a}}\underline{\phantom{a}}\underline{\phantom{a}}\underline{\phantom{a}}\underline{\phantom{a}}\underline{\phantom{a}}\underline{\phantom{a}}\underline{\phantom{a}}\underline{\phantom{a}}\underline{\phantom{a}}\underline{\phantom{a}}\underline{\phantom{a}}\underline{\phantom{a}}\underline{\phantom{a}}\underline{\phantom{a}}\underline{\phantom{a}}\underline{\phantom{a}}\underline{\phantom{a}}\underline{\phantom{a}}\underline{\phantom{a}}\underline{\phantom{a}} 
\end{enumerate} 


{\fontsize{14}{4}\selectfont \textbf{Дополнительное задание:}} \\
\begin{tikzpicture}[shorten >=1pt,node distance=3cm,on grid,auto]
\begin{center}
   \node[state] (b1) {$b_1$};
   \node[state] (b2) [right=of b1] {$b_2$};
   \node[state] (b3) [right=of b2] {$b_3$};
   \node[state] (b4) [below=of b1] {$b_4$};
   \node[state] (b5) [below=of b2] {$b_5$};
   \node[state] (b6) [below=of b3] {$b_6$};
   
   \path[->]
    (b1) edge [loop left] node {$z_1$} (b1)
    (b1) edge node {$z_3$} (b2)
    (b2) edge node {$z_1$} (b6)
    (b6) edge [->] node {$z_3$} (b3)
    (b3) edge node {$z_2$} (b2)
    (b3) edge [bend left] node {$z_1$} (b6)
    (b2) edge [loop above] node {$z_2$} (b2)
    (b1) edge node {$z_2$} (b5)
    (b5) edge node {$z_2$} (b4)
    (b6) edge [bend left] node {$z_2$} (b4)
    (b4) edge [loop left] node {$z_3$} (b4)
    (b4) edge node {$z_1$} (b1)
    (b5) edge node {$z_2$} (b3);
\end{center}
\end{tikzpicture}
\end{document}
